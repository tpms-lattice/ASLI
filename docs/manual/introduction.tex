\section{Introduction} \label{sec:intro}
\asli{} (A Simple Lattice Infiller) is a cross-platform command line open-source tool written in C++ that gives users the ability to provide functionally graded lattice infills to any 3D geometry. The lattice infill is constructed out of unit cells, described by implicit functions, whose type, size and feature can be varied locally to obtain the desired local properties. \asli{} also has an optional Graphical User Interface (GUI) available, called \qasli{}.

Details on the technical aspects of \asli{} can be found in ``\href{https://doi.org/10.1080/17452759.2022.2048956}{A flexible and easy-to-use open-source tool for designing functionally graded 3D porous structures}''.

\subsection{Authors} \label{sec:authors}
\asli{} and \qasli{} are developed by the \href{http://www.biomech.ulg.ac.be}{Biomechanics Research Unit} (University of Li\`{e}ge and KU Leuven). %Both are currently maintained by their principal developers.

\paragraph*{Principal developers}
\begin{itemize}
	\item F. Perez-Boerema [Developer of \asli{}] (KU Leuven, Belgium)
	\item M. Barzegari [Developer of \qasli{}] (KU Leuven, Belgium)
\end{itemize}

\paragraph*{Principal investigator}
\begin{itemize}
	\item L. Geris (University of Li\`{e}ge and KU Leuven, Belgium)
\end{itemize}

\subsection{Licensing} \label{sec:licensing}
\asli{} is licensed under the terms of the GNU Affero General Public License.

\subsection{Citing \asli{}}
If you use \asli{} for your research we kindly ask you to cite:
\begin{itemize}
	\item F. Perez-Boerema, M. Barzegari and L. Geris. (2022). A flexible and easy-to-use open-source tool for designing functionally graded 3D porous structures, \textit{Virtual and Physical Prototyping}, 17:3, 682-699, DOI: \href{https://doi.org/10.1080/17452759.2022.2048956}{\seqsplit{10.1080/17452759.2022.2048956}}.
	
\end{itemize}

\subsection{Acknowledgments}
The authors gratefully acknowledge funding from the European Regional Development Fund – Interreg VA Flanders - The Netherlands (PRosPERoS, CCI 2014\-TC\-16\-RFCB\-046), the European Union’s Horizon 2020 research and innovation programme via the European Research Council (ERC CoG INSITE 772418), the Fund for Scientific Research Flanders (G085018N), the Fédération Wallonie-Bruxelles through the BioWin project BIOPTOS (7560) and the KU Leuven Special Research Fund (C24/17/07).